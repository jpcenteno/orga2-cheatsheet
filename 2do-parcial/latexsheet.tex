\documentclass[10pt,landscape]{article}
\usepackage{multicol}
\usepackage{calc}
\usepackage{ifthen}
\usepackage[landscape]{geometry}
\usepackage{hyperref}
\usepackage{booktabs}

% Template clonado de https://wch.github.io/latexsheet/. Licencia 
% To make this come out properly in landscape mode, do one of the following
% 1.
%  pdflatex latexsheet.tex
%
% 2.
%  latex latexsheet.tex
%  dvips -P pdf  -t landscape latexsheet.dvi
%  ps2pdf latexsheet.ps


% If you're reading this, be prepared for confusion.  Making this was
% a learning experience for me, and it shows.  Much of the placement
% was hacked in; if you make it better, let me know...


% 2008-04
% Changed page margin code to use the geometry package. Also added code for
% conditional page margins, depending on paper size. Thanks to Uwe Ziegenhagen
% for the suggestions.

% 2006-08
% Made changes based on suggestions from Gene Cooperman. <gene at ccs.neu.edu>


% To Do:
% \listoffigures \listoftables
% \setcounter{secnumdepth}{0}


% This sets page margins to .5 inch if using letter paper, and to 1cm
% if using A4 paper. (This probably isn't strictly necessary.)
% If using another size paper, use default 1cm margins.
\ifthenelse{\lengthtest { \paperwidth = 11in}}
    { \geometry{top=.5in,left=.5in,right=.5in,bottom=.5in} }
    {\ifthenelse{ \lengthtest{ \paperwidth = 297mm}}
        {\geometry{top=1cm,left=1cm,right=1cm,bottom=1cm} }
        {\geometry{top=1cm,left=1cm,right=1cm,bottom=1cm} }
    }

% Turn off header and footer
\pagestyle{empty}
 

% Redefine section commands to use less space
\makeatletter
\renewcommand{\section}{\@startsection{section}{1}{0mm}%
                                {-1ex plus -.5ex minus -.2ex}%
                                {0.5ex plus .2ex}%x
                                {\normalfont\large\bfseries}}
\renewcommand{\subsection}{\@startsection{subsection}{2}{0mm}%
                                {-1explus -.5ex minus -.2ex}%
                                {0.5ex plus .2ex}%
                                {\normalfont\normalsize\bfseries}}
\renewcommand{\subsubsection}{\@startsection{subsubsection}{3}{0mm}%
                                {-1ex plus -.5ex minus -.2ex}%
                                {1ex plus .2ex}%
                                {\normalfont\small\bfseries}}
\makeatother

% Define BibTeX command
\def\BibTeX{{\rm B\kern-.05em{\sc i\kern-.025em b}\kern-.08em
    T\kern-.1667em\lower.7ex\hbox{E}\kern-.125emX}}

% Don't print section numbers
\setcounter{secnumdepth}{0}


\setlength{\parindent}{0pt}
\setlength{\parskip}{0pt plus 0.5ex}

% Column separator
\setlength{\columnseprule}{0.25pt}

% -----------------------------------------------------------------------

\begin{document}

\raggedright
\footnotesize
\begin{multicols*}{3}


% multicol parameters
% These lengths are set only within the two main columns
%\setlength{\columnseprule}{0.25pt}
\setlength{\premulticols}{1pt}
\setlength{\postmulticols}{1pt}
\setlength{\multicolsep}{1pt}
\setlength{\columnsep}{2pt}

% Titulo.
\begin{center}
    \Large{\textbf{ORGA-II Cheatsheet. \\
                   2\textsuperscript{do} parcial.} } \\
\end{center}

% ------------------------------------------------------------------------------

\section{Registros de uso general.}

\begin{tabular}{l l l l l l}
    \toprule
    \textbf{4 Bytes} & \textbf{2 Bytes} & \textbf{1 Byte} & \textbf{ABI C} \\
    \textit{(dword)} & \textit{(word)} & \textit{(byte)} & \\
    \midrule
    \texttt{eax} & \texttt{ax} & \texttt{al} & return 1\textsuperscript{ro} \\
    \texttt{ebx} & \texttt{bx} & \texttt{bl} & preservado \\
    \texttt{ecx} & \texttt{cx} & \texttt{cl} & --- \\
    \texttt{edx} & \texttt{dx} & \texttt{dl} & return 2\textsuperscript{do} \\
    \texttt{esi} & \texttt{si} & \texttt{sil} & preservado \\
    \texttt{edi} & \texttt{di} & \texttt{dil} & preservado \\
    \texttt{ebp} & \texttt{bp} & \texttt{bpl} & preservado \\
    \texttt{esp} & \texttt{sp} & \texttt{spl} & preservado \\
    \bottomrule
\end{tabular}

% ------------------------------------------------------------------------------

\section{Orden de parametros PUSHAD}%
\label{sec:orden_de_parametros_pushad}

La instruccion \texttt{PUSHAD} pushea los registros de uso general en el
siguiente orden:

\begin{tabular}{l l l l l l}
    \toprule
    \textbf{Registro} & \textbf{Offset} \\
    \midrule
    \texttt{EAX} & $\mathtt{ESP} +28 $ \\
    \texttt{ECX} & $\mathtt{ESP} +24 $ \\
    \texttt{EDX} & $\mathtt{ESP} +20 $ \\
    \texttt{EBX} & $\mathtt{ESP} +16 $ \\
    \texttt{ESP} & $\mathtt{ESP} +12 $ \\
    \texttt{EBP} & $\mathtt{ESP} +8 $ \\
    \texttt{ESI} & $\mathtt{ESP} +4 $ \\
    \texttt{EDI} & $\mathtt{ESP} +0 $ \\
    \bottomrule
\end{tabular}

% ------------------------------------------------------------------------------

\section{Patrones en Assembler.}


\subsection{Ciclo While.}

\begin{verbatim}
        jmp .cond ; Directo a la guarda
.loop:  ...       ; Cuerpo del loop
.cond:  cmp ax,1  ; Guarda (Ej. ax == 1)
        je .loop  ; Jump al cuerpo del loop.
\end{verbatim}

\subsection{Ciclo Do While.}

Para el ciclo \textbf{do while} sacamos el primer jump a la condicion. No es
necesaria la etiqueta \texttt{.cond}:

\begin{verbatim}
.loop:  ...      ; Cuerpo del loop
        cmp ax,1 ; Guarda (Ej. ax == 1)
        je .loop ; Jump al cuerpo del loop.
\end{verbatim}

\subsection{Ciclo For.}

Para un ciclo \texttt{for n \ldots n} usamos el orden inverso de
\texttt{for (i=0; i < n; ++i)} que equivale a \texttt{for (i=n; 0 < i; --i)}:

\begin{verbatim}
        mov cx, n   ; cx = n
        jmp .cond   ; Jump to the condition
.loop:  ...         ; Execute the content of the loop
.cond:  cmp cx, 0   ; Check that 0 < cx
        jg .loop
\end{verbatim}

\end{multicols}

\end{document}
