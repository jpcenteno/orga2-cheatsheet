\documentclass[10pt,landscape]{article}
\usepackage{multicol}
\usepackage{calc}
\usepackage{ifthen}
\usepackage[landscape]{geometry}
\usepackage{hyperref}
\usepackage{booktabs}

% Template clonado de https://wch.github.io/latexsheet/. Licencia 
% To make this come out properly in landscape mode, do one of the following
% 1.
%  pdflatex latexsheet.tex
%
% 2.
%  latex latexsheet.tex
%  dvips -P pdf  -t landscape latexsheet.dvi
%  ps2pdf latexsheet.ps


% If you're reading this, be prepared for confusion.  Making this was
% a learning experience for me, and it shows.  Much of the placement
% was hacked in; if you make it better, let me know...


% 2008-04
% Changed page margin code to use the geometry package. Also added code for
% conditional page margins, depending on paper size. Thanks to Uwe Ziegenhagen
% for the suggestions.

% 2006-08
% Made changes based on suggestions from Gene Cooperman. <gene at ccs.neu.edu>


% To Do:
% \listoffigures \listoftables
% \setcounter{secnumdepth}{0}


% This sets page margins to .5 inch if using letter paper, and to 1cm
% if using A4 paper. (This probably isn't strictly necessary.)
% If using another size paper, use default 1cm margins.
\ifthenelse{\lengthtest { \paperwidth = 11in}}
    { \geometry{top=.5in,left=.5in,right=.5in,bottom=.5in} }
    {\ifthenelse{ \lengthtest{ \paperwidth = 297mm}}
        {\geometry{top=1cm,left=1cm,right=1cm,bottom=1cm} }
        {\geometry{top=1cm,left=1cm,right=1cm,bottom=1cm} }
    }

% Turn off header and footer
\pagestyle{empty}
 

% Redefine section commands to use less space
\makeatletter
\renewcommand{\section}{\@startsection{section}{1}{0mm}%
                                {-1ex plus -.5ex minus -.2ex}%
                                {0.5ex plus .2ex}%x
                                {\normalfont\large\bfseries}}
\renewcommand{\subsection}{\@startsection{subsection}{2}{0mm}%
                                {-1explus -.5ex minus -.2ex}%
                                {0.5ex plus .2ex}%
                                {\normalfont\normalsize\bfseries}}
\renewcommand{\subsubsection}{\@startsection{subsubsection}{3}{0mm}%
                                {-1ex plus -.5ex minus -.2ex}%
                                {1ex plus .2ex}%
                                {\normalfont\small\bfseries}}
\makeatother

% Define BibTeX command
\def\BibTeX{{\rm B\kern-.05em{\sc i\kern-.025em b}\kern-.08em
    T\kern-.1667em\lower.7ex\hbox{E}\kern-.125emX}}

% Don't print section numbers
\setcounter{secnumdepth}{0}


\setlength{\parindent}{0pt}
\setlength{\parskip}{0pt plus 0.5ex}

% Column separator
\setlength{\columnseprule}{0.25pt}

% -----------------------------------------------------------------------

\begin{document}

\raggedright
\footnotesize
\begin{multicols*}{3}


% multicol parameters
% These lengths are set only within the two main columns
%\setlength{\columnseprule}{0.25pt}
\setlength{\premulticols}{1pt}
\setlength{\postmulticols}{1pt}
\setlength{\multicolsep}{1pt}
\setlength{\columnsep}{2pt}

% Titulo.
\begin{center}
    \Large{\textbf{ORGA-II --- 2\textsuperscript{do} parcial.} } \\
\end{center}

% ------------------------------------------------------------------------------

\section{Registros de uso general.}

\begin{tabular}{l l l l l l}
    \toprule
    \textbf{4 Bytes} & \textbf{2 Bytes} & \textbf{1 Byte} & \textbf{ABI C} \\
    \textit{(dword)} & \textit{(word)} & \textit{(byte)} & \\
    \midrule
    \texttt{eax} & \texttt{ax} & \texttt{al} & return 1\textsuperscript{ro} \\
    \texttt{ebx} & \texttt{bx} & \texttt{bl} & preservado \\
    \texttt{ecx} & \texttt{cx} & \texttt{cl} & --- \\
    \texttt{edx} & \texttt{dx} & \texttt{dl} & return 2\textsuperscript{do} \\
    \texttt{esi} & \texttt{si} & \texttt{sil} & preservado \\
    \texttt{edi} & \texttt{di} & \texttt{dil} & preservado \\
    \texttt{ebp} & \texttt{bp} & \texttt{bpl} & preservado \\
    \texttt{esp} & \texttt{sp} & \texttt{spl} & preservado \\
    \bottomrule
\end{tabular}

% ------------------------------------------------------------------------------

\section{Orden de parametros PUSHAD}%
\label{sec:orden_de_parametros_pushad}

La instruccion \texttt{PUSHAD} pushea los registros de uso general en el
siguiente orden:

\begin{tabular}{l l l l l l}
    \toprule
    \textbf{Registro} & \textbf{Offset} \\
    \midrule
    \texttt{EAX} & $\mathtt{ESP} +28 $ \\
    \texttt{ECX} & $\mathtt{ESP} +24 $ \\
    \texttt{EDX} & $\mathtt{ESP} +20 $ \\
    \texttt{EBX} & $\mathtt{ESP} +16 $ \\
    \texttt{ESP} & $\mathtt{ESP} +12 $ \\
    \texttt{EBP} & $\mathtt{ESP} +8 $ \\
    \texttt{ESI} & $\mathtt{ESP} +4 $ \\
    \texttt{EDI} & $\mathtt{ESP} +0 $ \\
    \bottomrule
\end{tabular}

% ------------------------------------------------------------------------------

\section{Patrones en Assembler.}


\subsection{Ciclo While.}

\begin{verbatim}
        jmp .cond ; Directo a la guarda
.loop:  ...       ; Cuerpo del loop
.cond:  cmp ax,1  ; Guarda (Ej. ax == 1)
        je .loop  ; Jump al cuerpo del loop.
\end{verbatim}

\subsection{Ciclo Do While.}

Para el ciclo \textbf{do while} sacamos el primer jump a la condicion. No es
necesaria la etiqueta \texttt{.cond}:

\begin{verbatim}
.loop:  ...      ; Cuerpo del loop
        cmp ax,1 ; Guarda (Ej. ax == 1)
        je .loop ; Jump al cuerpo del loop.
\end{verbatim}

\subsection{Ciclo For.}

Para un ciclo \texttt{for n \ldots n} usamos el orden inverso de
\texttt{for (i=0; i < n; ++i)} que equivale a \texttt{for (i=n; 0 < i; --i)}:

\begin{verbatim}
        mov cx, n   ; cx = n
        jmp .cond   ; Jump to the condition
.loop:  ...         ; Execute the content of the loop
.cond:  cmp cx, 0   ; Check that 0 < cx
        jg .loop
\end{verbatim}

% ------------------------------------------------------------------------------

\section{Componentes de una direcci\'on lineal.}%
\label{sec:componentes_de_una_direcci'on_lineal_}

\begin{verbatim}
#define LIN_IX_DIR(lin)   ((lin >> 22) & 0x3FF)
#define LIN_IX_TAB(lin)   ((lin >> 12) & 0x3FF)
#define LIN_OFFSET(lin)   (lin & 0xFFF)
#define ADDR_PAGE(addr)   (phy & (~0x3FF))
#define ADDR_OFFSET(addr) (phy & 0x3FF)
\end{verbatim}

\section{Mapear P\'agina.}%
\label{sec:mapear_p'agina_}

\begin{verbatim}
void mmu_mapPage(page_dir_entry_t* cr3,
                 uint32_t lin,
                 uint32_t phy,
                 uint32_t attrs) {
    uint32_t dir_ix = LIN_IX_DIR(lin);
    uint32_t tab_ix = LIN_IX_TAB(lin);
    if (! cr3[dir_ix].p ) {
        dir[dir_ix].p = 1
        dir[dir_ix].rw = 1;
        dir[dir_ix].us = 1;
        dir[dir_ix].pwt = 0;
        dir[dir_ix].pcd = 0;
        dir[dir_ix].a = 0;
        dir[dir_ix].ign = 0;
        dir[dir_ix].ps = 0; // 4k
        dir[dir_ix].g = 0;
        dir[dir_ix].avl = 0;
        mmu_table_entry_t* table =
            mmu_getFreeKernelPage();
        dir[dir_ix].table_addr = table >> 12;
        for (i = 0; i < 1024; ++i){
            table[i].p = 0;
        }
    }
    uint32_t* table = dir[dir_ix] << 12;
    table[tab_ix] = ADDR_PAGE(phy) | attrs;
    tlb_flush(); // Borra cache paginacion
}
\end{verbatim}


% ------------------------------------------------------------------------------

\section{Scheduler}%
\label{sec:scheduler}

\begin{verbatim}
offset:   dd 0
selector: dw 0
global _isrClock
_isrClock:
    pushad
    call pic_finish
    call schedNextTask
    str cx
    cmp ax, cx
    je .fin % No puede saltar a la misma xq bussy
    mov [selector], ax
    jmp far [offset]
    % Aca entra al volver a la tarea.
.fin:
    popad
    iret
\end{verbatim}

\section{proteccion}%
\label{sec:proteccion}

\subsection{Chequeos de segmentacion}%
\label{sub:segmentacion}

\begin{itemize}
    \item \textbf{DPL} Descriptor Privilege Level. Nivel de privilegio del
           segmento a ser accedido.
    \item \textbf{CPL} Current Privilege Level. Es el nivel de privilegio del
          \texttt{CS} que estoy ejecutando.
    \item \textbf{RPL} Nivel de privilegio indicado en el selector de segmento.
    \item \textbf{EPL}
          $\texttt{EPL} = \texttt{maxNumerico}(\texttt{CPL},\texttt{RPL})$
\end{itemize}
Reglas: (Si no se cumplen, tira \texttt{\#GP})
\begin{itemize}
    \item Para \textbf{acceder a un segmento de datos} necesitamos $\mathtt{EPL} \leq \texttt{DPL}$.
    \item Para \textbf{acceder a un segmento de codigo} necesitamos:
    \begin{itemize}
        \item Si el segmento es \textbf{non-conforming}, $\texttt{EPL} = \texttt{DPL}$
        \item Si el segmento es \textbf{conforming}, $\texttt{CPL} \geq \texttt{DPL}$
    \end{itemize}
\end{itemize}

\begin{tabular}{l l l l l l}
    \toprule
    \textbf{Operacion} & \textbf{Bit S} & \textbf{Datos} & \textbf{Codigo}\\
    \midrule
    Leer     & 1 & Siembre      & R habilitado    \\
    Escribir & 1 & W habilitado & \#GP            \\
    Ejecutar & 1 & \#GP         & Si (ver reglas) \\
    \bottomrule
\end{tabular}

\subsection{chequeos de paginacion}%
\label{sub:chequeos_de_paginacion}

\begin{itemize}
    \item Primero chequea el bit presente de la tabla.
    \item despues chequea el bit P de la pagina.
    \item Si es operacion de escritura chequea el bit \texttt{R/W}
    \item Verifica el bit \texttt{U/S}. Si \texttt{US=0}, solo se puede acceder
          en nivel 0.
\end{itemize}

De no cumplirse un chequeo de paginacion, se genera un \texttt{\#PF}.

\subsection{Chequeos en interrupciones}%
\label{sub:chequeos_en_interrupciones}

\begin{itemize}
    \item Primero se verifica que el descriptor de la \texttt{IDT} tenga su bit
          \texttt{P=1}.
    \item Se verifica que $ \texttt{CPL} \leq \texttt{IDT[i].DPL} $. Si la
          interrupcion es de hardware, los bits de la \texttt{DPL} se ignoran.
\end{itemize}

De no cumplirse el chequeo, se genera un \texttt{\#PF}.

\subsection{Chequeos al cambiar de tarea}%
\label{sub:chequeos_al_cambiar_de_tarea}

\begin{itemize}
    \item Se verifica que tenga el bit \texttt{P=1} en su descriptor de
          \texttt{tss}.
    \item Se verifica que tenga el bit \texttt{busy=0} en la \texttt{tss}.
\end{itemize}

De no cumplirse el chequeo, se genera un \texttt{\#PF}.

% ------------------------------------------------------------------------------

\section{Macros para direcciones}%
\label{sec:macros}

\begin{verbatim}
#define PD_PHYS_DIR(cr3) ((pde_t*) (((uint32) cr3) & 0xFFFFF000))
#define PT_PHYS_ADDR(pde) ((pte_t*) (pde.table_addr << 12))
#define PAGE_PHYS_ADDR(pte) (pte.page_addr << 12)
\end{verbatim}

% ------------------------------------------------------------------------------

\section{TSS desc init}%
\label{sec:tss_gdt}

\begin{verbatim}
void tss_initGdtEntry(uint32_t ix,
    tss* base, uint32_t dpl) {
    gdt[ix].limit_0_15   = TSS_SIZE;
    gdt[ix].base_0_15    = (uint32) base & 0xFFFF;
    gdt[ix].base_23_16   = ((uint32) base >> 16) & 0xFF;
    gdt[ix].type         = 0b1001;
    gdt[ix].s            = 0x0;
    gdt[ix].dpl          = dpl;
    gdt[ix].p            = 0x1;
    gdt[ix].limit_16_19  = 0x0;
    gdt[ix].avl          = 0x0;
    gdt[ix].l            = 0x0;
    gdt[ix].db           = 0x0;
    gdt[ix].g            = 0x0;
    gdt[ix].base_31_24   = ((uint32_t) base >> 24);
}
\end{verbatim}

% ------------------------------------------------------------------------------

\section{TSS init}%
\label{sec:tss}

\begin{verbatim}
void fill_tss(tss* actual) {
    actual->eip = TASK_CODE;
    actual->cr3 = (uint32_t) mmu_initTaskDir();
    actual->eflags  = 0x00000202;
    actual->iomap = 0xFFFF;
    actual->cs = GDT_OFF_CODIGO_USER;
    actual->gs = GDT_OFF_DATOS_USER;
    actual->fs = GDT_OFF_DATOS_USER;
    actual->ds = GDT_OFF_DATOS_USER;
    actual->es = GDT_OFF_DATOS_USER;
    actual->ss = GDT_OFF_DATOS_USER;
    actual->eax = 0x0;
    actual->ecx = 0x0;
    actual->edx = 0x0;
    actual->ebx = 0x0;
    actual->esi = 0x0;
    actual->edi = 0x0;
    actual->esp = TASK_STACK;
    actual->ebp = TASK_STACK;
    actual->esp0 = mmu_nextFreeKernelPage() + PAGE_SIZE;
    actual->ss0  = GDT_OFF_DATOS_KERNEL;
}
\end{verbatim}


\end{multicols*}

\end{document}
