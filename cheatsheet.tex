\documentclass{article}

\usepackage{booktabs}

\begin{document}
\title{Machete de Orga2}
\author{Joaqu\'in P. Centeno}
\maketitle

\section{Registros}

\begin{tabular}{l l l l l l}
    \toprule
    \textbf{8 Bytes} & \textbf{4 Bytes} & \textbf{2 Bytes} & \textbf{1 Byte} & \textbf{ABI C} & \textbf{Preservado} \\
    \midrule
    \texttt{rax} & \texttt{eax}  & \texttt{ax}   & \texttt{al}   & Return       &    \\
    \texttt{rbx} & \texttt{ebx}  & \texttt{bx}   & \texttt{bl}   &              & si \\
    \texttt{rcx} & \texttt{ecx}  & \texttt{cx}   & \texttt{cl}   & 4th argument &    \\
    \texttt{rdx} & \texttt{edx}  & \texttt{dx}   & \texttt{dl}   & 3rd argument &    \\
    \texttt{rsi} & \texttt{esi}  & \texttt{si}   & \texttt{sil}  & 2nd argument &    \\
    \texttt{rdi} & \texttt{edi}  & \texttt{di}   & \texttt{dil}  & 1st argument &    \\
    \texttt{rbp} & \texttt{ebp}  & \texttt{bp}   & \texttt{bpl}  & Stack        & si \\
    \texttt{rsp} & \texttt{esp}  & \texttt{sp}   & \texttt{spl}  & Stack        & si \\
    \texttt{r8}  & \texttt{r8d}  & \texttt{r8w}  & \texttt{r8b}  & 5th argument &    \\
    \texttt{r9}  & \texttt{r9d}  & \texttt{r9w}  & \texttt{r9b}  & 6th argument &    \\
    \texttt{r10} & \texttt{r10d} & \texttt{r10w} & \texttt{r10b} &              &    \\
    \texttt{r11} & \texttt{r11d} & \texttt{r11w} & \texttt{r11b} &              &    \\
    \texttt{r12} & \texttt{r12d} & \texttt{r12w} & \texttt{r12b} &              & si \\
    \texttt{r13} & \texttt{r13d} & \texttt{r13w} & \texttt{r13b} &              & si \\
    \texttt{r14} & \texttt{r14d} & \texttt{r14w} & \texttt{r14b} &              & si \\
    \texttt{r15} & \texttt{r15d} & \texttt{r15w} & \texttt{r15b} &              & si \\
    \bottomrule
\end{tabular}

\section{Patrones}


\subsection{Ciclo \texttt{while}}

\begin{verbatim}
        jmp .cond   ; Jump to condition first
.loop:  ...         ; Execute the content of the loop
.cond:  cmp ax,1    ; Check the condition (i.e. ax == 1)
        je .loop    ; Jump to content of the loop if met
\end{verbatim}

\subsection{Ciclo \texttt{do while}}

Para el ciclo \textbf{do while} sacamos el primer jump a la condicion. No es
necesaria la etiqueta \texttt{.cond}:

\begin{verbatim}
.loop:  ...         ; Whatever you wanna do goes here
        cmp ax,1    ; Check the condition (i.e. ax == 1)
        je .loop    ; And loop if equal
\end{verbatim}

\subsection{Ciclo \texttt{for}}

Para un ciclo \texttt{for n...n} usamos el orden inverso de
\texttt{for(i=0; i < n; ++i)} que equivale a \texttt{for(i=n; 0 < i; --i)}:

\begin{verbatim}
        mov cx, n   ; cx = n
        jmp .cond   ; Jump to the condition
.loop:  ...         ; Execute the content of the loop
.cond:  cmp cx, 0   ; Check that 0 < cx
        jg .loop
\end{verbatim}

\subsection{Hack para cargar valor de un registro en FPU}

No existe forma de cargar un dato a la \textit{FPU} desde de un registro sin
pasar por la memoria. Para ahorrarnos el \texttt{malloc} podemos escribir en una
parte no inicializada (parte negativa) del stack.

\begin{verbatim}
mov [rsp - 4], eax   ; Donde EAX tiene un entero signed
fild DWORD [rsp - 4] ; Convierte a float y pushea al stack
\end{verbatim}


\section{GDB}

\subsection{imprimir una posici\'on de memoria}

El comando \texttt{x} permite explorar una posicion en memoria. \texttt{n} es
la cantidad de unidades. \texttt{u} es el tamano en bytes de la unidad y
\texttt{f} es el formato para expresar la unidad.

\begin{verbatim}
(gdb) x/nfu <address>
\end{verbatim}

\begin{tabular}{l l l}
    \toprule
    \textbf{unit} & \textbf{Description} & \textbf{\# Bytes} \\
    \midrule
    \texttt{b} & \textit{Byte}     & 1\\
    \texttt{h} & \textit{HalfWord} & 2\\
    \texttt{w} & \textit{Word}     & 4\\
    \texttt{g} & \textit{GiantWord}& 8\\
    \bottomrule
\end{tabular}

\begin{tabular}{l l}
    \toprule
    \textbf{format} & \textbf{Description} \\
    \midrule
    \texttt{a} & Pointer \\
    \texttt{c} & Read as integer, print as character \\
    \texttt{d} & Integer, signed decimal \\
    \texttt{f} & Floating point number \\
    \texttt{o} & Integer, print as octal \\
    \texttt{s} & Try to treat as C string \\
    \texttt{t} & Integer, print as binary (t = `two') \\
    \bottomrule
\end{tabular}


\subsection{watchpoints}
Poner un watchpoint en una posicion de memoria:
\begin{verbatim}
(gdb) watch *0x40a5d8
\end{verbatim}

\section{SIMD}

\subsection{Empaquetado de datos}

\begin{tabular}{l l l l}
    \toprule
    \textbf{Size Dato} & \textbf{nombre C} & \textbf{Nombre INTEL} & \textbf{Cantidad} \\
    \midrule
    1 Byte & \texttt{char}   & \texttt{byte}     & 16 \\
    2 Byte & \texttt{short}  & \texttt{word}     & 8  \\
    4 Byte & \texttt{int}    & \texttt{Double}   & 4  \\
    8 Byte & \texttt{long}   & \texttt{quadword} & 2  \\
    \midrule
    4 Byte & \texttt{float}  & \texttt{single}   & 4  \\
    8 Byte & \texttt{double} & \texttt{double}   & 2  \\
    \bottomrule
\end{tabular}

\end{document}
